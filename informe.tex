\documentclass{article}
\usepackage[utf8]{inputenc}
\usepackage{graphicx}
\usepackage[simplified]{pgf-umlcd}
\usepackage{tikz}
\usepackage{multirow}
\usepackage{float}
\usetikzlibrary{positioning,fit,calc,arrows.meta, shapes}
\usepackage{wrapfig}
\graphicspath{ {images/} }

%Tot això hauria d'anar en un pkg, però no sé com és fa
\newcommand*{\assignatura}[1]{\gdef\1assignatura{#1}}
\newcommand*{\grup}[1]{\gdef\3grup{#1}}
\newcommand*{\professorat}[1]{\gdef\4professorat{#1}}
\renewcommand{\tablename}{Taula}
\renewcommand{\title}[1]{\gdef\5title{#1}}
\renewcommand{\author}[1]{\gdef\6author{#1}}
\renewcommand{\date}[1]{\gdef\7date{#1}}
\renewcommand{\maketitle}{ %fa el maketitle de nou
    \begin{titlepage}
        \raggedright{UNIVERSITAT DE LLEIDA \\
            Escola Politècnica Superior \\
            Grau en Enginyeria Informàtica\\
            \1assignatura\\}
            \vspace{5cm}
            \centering\huge{\5title \\}
            \vspace{3cm}
            \large{\6author} \\
            \normalsize{\3grup}
            \vfill
            Professorat : \4professorat \\
            Data : \7date
\end{titlepage}}
%Emplenar a partir d'aquí per a fer el títol : no se com es fa el package
%S'han de renombrar totes, inclús date, si un camp es deixa en blanc no apareix

\tikzset{
	%Style of nodes. Si poses aquí un estil es pot reutilitzar més facilment
	base/.style = {circle, draw=black,
      minimum width=0.75cm, font=\ttfamily,
      text centered},
    dots/.style = {circle, draw=white,
      minimum width=0.75cm, font=\ttfamily,
      text centered},
    last/.style = {base, fill=orange!15},
    remove/.style = {base, fill=red!15},
    change/.style = {base, fill=green!15},
    tree/.style = {base, rectangle, minimum height=0.75cm},
    stack/.style = {rectangle, font=\ttfamily, rounded corners, draw=black,
      minimum width=4cm, minimum height=1cm,
      text centered},
   	even/.style = {stack, fill=green!30},
   	odd/.style = {stack, fill=orange!15},
   	blank/.style = {stack, minimum height=0.5cm, draw=white},
   	typetag/.style={rectangle, draw=black!50, font=\ttfamily, anchor=west}
}
\renewcommand{\figurename}{Figura}
\title{Pràctica: Jerarquia de Memòria}
\author{Sergi Sales Jové, Sergi Simón Balcells}
\date{Dimecres 24 de Abril}
\assignatura{Arquitectura de Computadors}
\professorat{Concepció Roig}
\grup{GM3}

%Comença el document

\begin{document}
\maketitle
\thispagestyle{empty}

\newpage
\pagenumbering{roman}
\tableofcontents
\newpage
\pagenumbering{arabic}
\section{Introducció}
\section{Anàlisi de resultats}
% A nivell de cost de fallades, 1K / 16 / 4 és el millor
% A nivell de cost temporal, pot ser que 1K / 16 / 2 sigui millor, i és mica més barat
% Si no es pot permetre, i cost de comparados << a cost cache, 128 / 16 / 4.
% S el cost dels comparadors es substancial: 128 / 8 / 2

\subsection{Característiques de MC}
Els resultats que s'analitzaran es troba en la Taula \ref{tab:mc} del Annex.\\
\\
Si és un sistema crític que necessita la mínima quantitat de fallades per a ser més
ràpid, llavors el sistema amb \textbf{M}emòria \textbf{C}au (MC) hauria de ser de 1KB
de mida, la mida de bloc hauria de ser de 16 bytes, i hauria de tenir 4 blocs per conjunt.\\
\\
Si el que es busca és abaratir els costos, donat que el canvi de memòria és de 4 vegades més
alt, i els costos presumiblement s'incrementen linealment, llavors utilitzar 128 bytes, 16 bytes i
4 blocs per conjunt és una bona opció en comparació a la quantitat de fallades que tenen la MC amb
mida 64 bytes, així com el cost no és tan elevat com el que requereix una Memòria de 64 bytes,
i, el cost de millorar a dos blocs conjunts amb 128 bytes de mida no representa un augment tant gran
en comparació a la pèrdua de temps.\\
\\
Finalment, com a resultat òptim creiem que seria 1Kbyte de mida amb 16 bytes de mida de bloc i
dos blocs per conjunt. La millora que proporciona respecte a la MC presentada anteriorment de 128 bytes
és molt més substancial que l'increment del preu (el preu més o menys es multiplica per 4, mentre que
el nombre de fallades es redueix amb un factor de 10) i la millora respecte al sistema crític no
és tan substancial com per a cobrir els costos d'augmentar la mida del conjunt.
\subsection{Algoritmes de Subtitució}
Pel que fa als algoritmes de substitució, segons podem veure a la Taula \ref{tab:sub}, amb qualsevol
 configuració de mida de la MC i mida de bloc, podem veure que els millors resultats són els de l'algoritme LRU.
 Això es pot donar gràcies al fet que aquest algoritme aprofita el principi de localitat temporal, el qual estableix 
que els elements referenciats per un procés en un determinat instant de temps tendiran a ser referenciats de nou.\\
\section{Conclusió}
\newpage
\section{Annex}
\begin{table}[!h]
\centering
\begin{tabular}{ |c|c|c|c|c| }
\hline
Mida MC& Mida bloc& directa& 2 blocs/conjunt& 4 blocs/conjunt \\
\hline
\multirow{3}{4em}{64 bytes} & 2 bytes & 26,301 & 23,966 & 15,212 \\
& 8 bytes & 23,548 & 18, 359 & 10,394 \\
& 16 bytes & 37,252 & 22,728 & 17,756 \\
\hline
\multirow{3}{4em}{128 bytes} & 2 bytes & 13,263 & 12,203 & 11,979 \\
& 8 bytes & 10,742 & 8,793 & 7,401 \\
& 16 bytes & 13,356 & 9,357 & 6,310 \\
\hline
\multirow{3}{4em}{1 Kbyte} & 2 bytes & 2,219 & 1,786 & 1,802 \\
& 8 bytes & 1,485 & 0,974 & 0,982 \\
& 16 bytes & 1,500 & 0,642 & 0,626 \\
\hline
\end{tabular}
\caption{Característiques MC}
\label{tab:mc}
\end{table}

\begin{table}[!h]
\centering
\begin{tabular}{|c|c|c|c|c|c|}
\hline
Mida MC & Mida Bloc & LRU (\%) & aleatori (\%) & FIFO (\%) & LFU (\%) \\
\hline
\multirow{3}{4em}{64 bytes}     & 2 bytes   & 10,541   & 25,706        & 11,577    & 11,778   \\
         & 8 bytes   & 9,149    & 25,628        & 11,445    & 11,121   \\
         & 16 bytes  & 17,756   & 30,570        & 22,388    & 20,416   \\
\hline
\multirow{3}{4em}{128 bytes}    & 2 bytes   & 6,999    & 16,302        & 7,184     & 8,360    \\
         & 8 bytes   & 6,125    & 16,085        & 7,061     & 6,937    \\
         & 16 bytes  & 5,344    & 15,266        & 7,099     & 6,674    \\
\hline
\multirow{3}{4em}{1 Kbyte}      & 2 bytes   & 1,725    & 7,053         & 1,725     & 1,725    \\
         & 8 bytes   & 0,936    & 3,271         & 0,936     & 0,936    \\
         & 16 bytes  & 0,650    & 2,173         & 0,812     & 0,665   
\\
\hline
\end{tabular}
\caption{Algorisme de substitució}
\label{tab:sub}
\end{table}
\end{document}

